\documentclass{beamer}


\usepackage[math]{fontspec}
\usepackage{unicode-math}
\setmathfont{Asana-Math.otf}
\usepackage{fontenc}
\setmainfont{Droid Serif}

\mode<presentation>
{
\usetheme{PaloAlto}
%\useoutertheme{tree}
\usecolortheme{lily}
\useinnertheme{rectangles}
\setbeamercovered{transparent}
\setbeamertemplate{blocks}[rounded][shadow=true]
}

\begin{document}

\section{Analyse Fonctionnelle Avancée et Applications aux EDP}

\begin{frame}\frametitle{Analyse Fonctionnelle Avancée et Applications
    aux EDP}
  \begin{block}{Objectifs}
    \begin{itemize}
    \item Préparation du M2MF 2015-2016
    \item Acquisition du vocabulaire et des outils mathématiques nécessaires à
      l'analyse des équations aux dérivées partielles
    \end{itemize}
  \end{block}
  \begin{block}{Objets}
    Les espaces $H^s$
    \begin{equation*}
      H^s = \{ \}
    \end{equation*}
  \end{block}
\end{frame}

\section{Méthodes Numériques pour les EDP}
\begin{frame}{Méthodes Numériques pour les EDP}
  \begin{block}{Objectifs}
    \begin{itemize}
    \item Étude de la méthode des éléments finis qui propose un cadre
      général pour passer de formulations continues à discrètes
    \item le cadre théorique est donnée par le cours d'Analyse
      Fonctionnelle Avancée
    \end{itemize}
  \end{block}

\end{frame}
\end{document}
