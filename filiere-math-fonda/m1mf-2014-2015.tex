\documentclass{beamer}

\usepackage{amsmath,amsfonts}
\usepackage[math]{fontspec}
\setmathfont{Asana-Math.otf}
\usepackage{fontenc}
\setmainfont{Droid Serif}

\mode<presentation>
{
\usetheme{PaloAlto}
%\useoutertheme{tree}
\usecolortheme{lily}
\useinnertheme{rectangles}
\setbeamercovered{transparent}
\setbeamertemplate{blocks}[rounded][shadow=true]
}

\begin{document}

\begin{frame}
  Les slides sont disponibles sur
  \begin{center}
    \centerline{\url{https://github.com/cemosis/unistra.ufr.math}}
  \end{center}
\end{frame}

\section{Analyse Fonctionnelle Avancée et Applications aux EDP}

\begin{frame}\frametitle{Analyse Fonctionnelle Avancée et Applications
    aux EDP}
  \begin{block}{Objectifs}
    \begin{itemize}
    \item Préparation du M2MF 2015-2016
    \item Acquisition du vocabulaire et des outils mathématiques nécessaires à
      l'analyse des équations aux dérivées partielles
    \end{itemize}
  \end{block}
  \begin{block}{Objets}
    Étant donné $\Omega \subset \mathbb{R}^d, d=1,2,3$, les espaces $H^s(\Omega)$
    \begin{equation*}
      H^s(\Omega)=\{ u \in L^2(\Omega)~|~\forall\alpha\text{ tel que }|\alpha|\le s,~D^\alpha u\in L^2(\Omega)\}
    \end{equation*}
  \end{block}
  \begin{block}{Questions}
    \begin{itemize}
    \item Propriétés de ces espaces
    \item Applications aux EDP: cadre fonctionel pour montrer
      l'existence et unicité de solutions
    \end{itemize}
  \end{block}
\end{frame}

\section{Méthodes Numériques pour les EDP}

\begin{frame}{Méthodes Numériques pour les EDP}

Motivations ici pour les EDP (video/images)...

\end{frame}
\begin{frame}{Méthodes Numériques pour les EDP}
  \begin{block}{Objectifs}
    \begin{itemize}
    \item Étude mathématique et numérique de la méthode des éléments
      finis qui propose un cadre général pour passer de formulations
      continues à discrètes
    \item le cadre théorique est donnée par le cours d'Analyse
      Fonctionnelle Avancée
    \end{itemize}
  \end{block}
  \begin{block}{Questions}
    \begin{itemize}
    \item Existence et unicité de solution pour des problèmes
      elliptiques linéaires coercifs au niveaux continus et discrets
    \item Construction de fonctions de bases, dites élément fini
    \item Erreur d'interpolation et d'approximation en norme $L^2$ et $H^1$
    \item Implémentation de la méthode et Vérification numériques des théorèmes
    \end{itemize}
  \end{block}
\end{frame}
\end{document}
